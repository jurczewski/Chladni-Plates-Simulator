\documentclass{classrep}
\usepackage[utf8]{inputenc}
\frenchspacing

\usepackage{graphicx}
\usepackage[usenames,dvipsnames]{color}
\usepackage[hidelinks]{hyperref}

\usepackage{amsmath, amssymb, mathtools}

\usepackage{fancyhdr, lastpage}
\pagestyle{fancyplain}
\fancyhf{}
\renewcommand{\headrulewidth}{0pt}
\cfoot{\thepage\ / \pageref*{LastPage}}


\studycycle{Informatyka, studia dzienne, I st.}
\coursesemester{VII}

\coursename{Technologie symulacji komputerowych}
\courseyear{2019/2020}

\courseteacher{dr. inż. Jan Rogowski}
\coursegroup{wtorek, 16:00}

\author{%
  \studentinfo[210347@edu.p.lodz.pl]{Krzysztof Wierzbicki}{210347}\\
  \studentinfo[210209@edu.p.lodz.pl]{Bartosz Jurczewski}{210209}%
}

\title{Zadanie 1.: Symulacja wahadła z ciałem stałym sypkim }

\begin{document}
\maketitle
\thispagestyle{fancyplain}

\newpage

\section{Cel}

\section{Wprowadzenie}

\section{Opis implementacji}

\section{Materiały i metody}

\section{Wyniki}

\section{Dyskusja}

\section{Wnioski}

\begin{thebibliography}{0}
  \bibitem{l2short} T. Oetiker, H. Partl, I. Hyna, E. Schlegl.
    \textsl{Nie za krótkie wprowadzenie do systemu \LaTeX2e}, 2007, dostępny
    online.
\end{thebibliography}

\end{document}
