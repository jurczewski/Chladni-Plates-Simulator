\documentclass{classrep}
\usepackage[utf8]{inputenc}
\frenchspacing

\usepackage{graphicx}
\usepackage{amssymb}
\usepackage[usenames,dvipsnames]{color}
\usepackage[hidelinks]{hyperref}

\usepackage{amsmath, amssymb, mathtools}

\usepackage{fancyhdr, lastpage}
\pagestyle{fancyplain}
\fancyhf{}
\renewcommand{\headrulewidth}{0pt}

\newcommand{\Csh}{C{\lserif\#}}

\studycycle{Informatyka, studia dzienne, I st.}
\coursesemester{VII}

\coursename{Technologie symulacji komputerowych}
\courseyear{2019/2020}

\courseteacher{dr. inż. Jan Rogowski}
\coursegroup{wtorek, 16:00}

\author{%
  \studentinfo[210347@edu.p.lodz.pl]{Krzysztof Wierzbicki}{210347}\\
  \studentinfo[210209@edu.p.lodz.pl]{Bartosz Jurczewski}{210209}%
}

\title{Zadanie: Symulacja płytki Chladniego }

\begin{document}
\maketitle
\thispagestyle{fancyplain}

\newpage

\section{Wstęp}
Zadaniem tworzonej przez nas aplikacji i modelu jest badanie drgań stalowej płytki wykonanej ze sprężystego materiału. \\
W symulacji zmianie będą mogły podlegać takie parametry jak: częstotliwość drgań, rozmiar kwadratowej płytki.

\section{Opis układu}
Symulacja układu będzie odbywać się w przestrzeni dwuwymiarowej, gdzie będziemy badać naprężenia występujące w stalowej płytce. 

\section{Opis obiektów biorących udział w symulacji}
W naszej symulacji możemy wyróżnić jeden główny obiekt będący fundamentem zagadnienia które chcemy symulować. Jest to wprawiona w drgania stalowa płyta. Zakładamy, że jest ona wykonana z materiału o kształcie płaskiego kwadratu, długość boku którego jest parametrem wejściowym symulacji.

Drgania własne dwuwymiarowej membrany można opisać równaniem falowym Bernoulliego
\begin{equation}
\frac{1}{c^2}\frac{\partial^2 \Psi}{\partial t^2}-\nabla ^2 \Psi = 0,\quad
\Psi \in \Omega \prod \R,\quad
\Omega \subset,
\end{equation}
gdzie $c$ jest prędkością rozchodzenia a $\Omega$ ograniczoną przestrzenią rozważań w $\mathbb{R}^2$.
Założenie, że $\Psi(t, x, y) = v(t) \times u(x, y)$ daje nam dwa równania różniczkowe
\begin{equation}
\partial^2_t v+\lambda c^2v=0
\quad \mathrm{and} \quad
\nabla ^2 u +\lambda u=0
\end{equation}
z dodatnią stałą $\lambda$.
Rozwiązanie równania dla $v(t)$ ma postać $v(t)=a \cos{\omega t}+b \sin{\omega t}$, gdzie $\omega=c \sqrt{\lambda}$. Cząstkowe równanie różniczkowe dla $u(x, y)$ w (2) doprowadza nas do problemu wartości własnej Laplasjanu, który próbujemy rozwiązać.
Po rozbiciu rozważanej powierzchni na siatkę kwadratów, możemy skorzystać z metody elementów skończonych aby rozwiązać równanie 
\begin{equation}
\mathcal{S} = \frac{1}{2}\iint_\Omega [(\partial_x u)^2+(\partial_y u)^2-\lambda u^2]\mathrm{d}x\mathrm{d}y+\frac{1}{2}\int_{\partial \Omega} u^2\mathrm{d}s
\end{equation}
dla układu wartości własnych, który pozwala powiązać z macierz sztywności z macierzą masy, jednocześnie wyznaczając
amplitudę węzła siatki. 

\section{Uproszczenia}
W naszym modelu i symulacji przyjęliśmy kilka następujących uproszczeń:
\begin{itemize}
	\item Brak oporów ruchu.
	\item W rozpatrywanym przez nas przypadku materiał z którego wykonana jest rozpatrywana płyta jest jednorodny oraz izotropowy -- jego gęstość jest taka sama w każdym punkcie, a moduł Younga jest niezależny od kierunku.
	\item Parametry wejściowe symulacji można zmieniać podawać w zakresach przyjętych przez nas i zamieszczonych w tym sprawozdaniu.
\end{itemize}

\section{Środowisko i biblioteka graficzna}
Program zostanie zrealizowany w środowisku graficznym Unity\\ (\url{www.unity.com}) za pomocą języka do niego przeznaczonego - C\#.

\begin{thebibliography}{0}
  \bibitem{l2short} T. Oetiker, H. Partl, I. Hyna, E. Schlegl.
    \textsl{Nie za krótkie wprowadzenie do systemu \LaTeX2e}, 2007, dostępny
    online.
  \bibitem{l2short} T. Müller
    \textsl{Numerical Chladni figures}, 2013, \url{https://arxiv.org/pdf/1308.5523.pdf}
\end{thebibliography}

\end{document}
