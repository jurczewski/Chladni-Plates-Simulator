\documentclass{classrep}
\usepackage[utf8]{inputenc}
\frenchspacing

\usepackage{graphicx}
\usepackage[usenames,dvipsnames]{color}
\usepackage[hidelinks]{hyperref}

\usepackage{amsmath, amssymb, mathtools}

\usepackage{fancyhdr, lastpage}
\pagestyle{fancyplain}
\fancyhf{}
\renewcommand{\headrulewidth}{0pt}

\newcommand{\Csh}{C{\lserif\#}}

\studycycle{Informatyka, studia dzienne, I st.}
\coursesemester{VII}

\coursename{Technologie symulacji komputerowych}
\courseyear{2019/2020}

\courseteacher{dr. inż. Jan Rogowski}
\coursegroup{wtorek, 16:00}

\author{%
  \studentinfo[210347@edu.p.lodz.pl]{Krzysztof Wierzbicki}{210347}\\
  \studentinfo[210209@edu.p.lodz.pl]{Bartosz Jurczewski}{210209}%
}

\title{Zadanie 1.: Symulacja wahadła z ciałem stałym sypkim }

\begin{document}
\maketitle
\thispagestyle{fancyplain}

\newpage

\section{Wstęp}
Zadaniem tworzonej przez nas aplikacji i modelu jest badanie toru po którym poruszało się będzie wahadło z którego wysypuje się sypki materiał. \\
W symulacji zmianie będą mogły podlegać takie parametry jak: prędkość początkowa, początkowa masa, prędkość wysypywana się piasku, długość wahadła, początkowy kierunek ruchu i początkowe odchylenie wahadła.

\section{Opis układu}
Symulacja wahadła będzie odbywać się w przestrzeni trójwymiarowej, a wizualizacją jego ruchu będą wzory pojawiające się na dwuwymiarowej płaszczyźnie. 

\section{Opis obiektów biorących udział w symulacji}
W naszej symulacji możemy wyróżnić dwa główne obiekty, będą fundamentem zagadnienia które chcemy symulować. Są to: wahadło, pole grawitacyjne i płaszczyzna na której pojawiają się ślady z sypkiego materiału wypełniającego wahadło.

\subsection{Wahadło}

\subsection{Płaszczyzna}

\subsection{Pole grawitacyjne}

\section{Opis oddziaływań między obiektami}
Pole grawitacyjne w którym porusza się wahadło powoduje powstanie siły działającej na wahadło skierowanej w dół. Wartość tej siły zależna jest od masy wahadła w danej chwili. Sypki materiał wysypujący się z wahadła odkłada się na dwuwymiarowej płaszczyźnie nad którą porusza się wahadło. Umiejscowienie materiału na płaszczyźnie zależne jest od chwilowego położenia wahadła.  

\section{Uproszczenia}
W naszym modelu i symulacji przyjęliśmy kilka następujących uproszczeń:
\begin{itemize}
    \item Brak oporów ruchu.
    \item W rozpatrywanym przez nas przypadku pole grawitacyjne jest zawsze jednorodne.
    \item Parametry wejściowe symulacji można zmieniać podawać w zakresach przyjętych przez nas i zamieszczonych w tym sprawozdaniu.
\end{itemize}


\section{Środowisko i biblioteka graficzna}
Program zostanie zrealizowany w środowisku graficznym Unity\\ (\url{www.unity.com}) za pomocą języka do niego przeznaczonego - C\#.

\begin{thebibliography}{0}
  \bibitem{l2short} T. Oetiker, H. Partl, I. Hyna, E. Schlegl.
    \textsl{Nie za krótkie wprowadzenie do systemu \LaTeX2e}, 2007, dostępny
    online.
\end{thebibliography}

\end{document}
